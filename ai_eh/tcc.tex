\pdfoutput=1
%% For double-blind review submission, w/o CCS and ACM Reference (max submission space)
%\documentclass[sigplan,10pt,review,anonymous]{acmart}
%\settopmatter{printfolios=false,printccs=false,printacmref=false}
%% For double-blind review submission, w/ CCS and ACM Reference
%\documentclass[sigplan,review,anonymous]{acmart}\settopmatter{printfolios=true}
%% For single-blind review submission, w/o CCS and ACM Reference (max submission space)
%\documentclass[sigplan,review]{acmart}\settopmatter{printfolios=true,printccs=false,printacmref=false}
%% For single-blind review submission, w/ CCS and ACM Reference
%\documentclass[sigplan,review]{acmart}\settopmatter{printfolios=true}
%% For final camera-ready submission, w/ required CCS and ACM Reference
\documentclass[sigplan,nonacm]{acmart}\settopmatter{printfolios=false,printccs=false,printacmref=false}

%% Bibliography style
\bibliographystyle{acmart}

\begin{document}
  \title{Catholic Viewpoints on Thinking Machinery}
  \begin{abstract}
  The discovery of an efficient means of simulating mental processes raises a variety of philosophical and theological questions. In this paper, we survey historical Catholic contributions and contemporary attitudes towards intellectual automation. In constrast with predominant secular views, we argue the development of thinking machines is not only compatible with the Catholic faith, but offers a more nuanced understanding of the relation between spirit and flesh. We survey historical and contemporary Catholic perspectives on thinking machinery, and how they inform our understanding of intelligence and the limits of computation.
  \end{abstract}

  \author{Breandan Considine}
  \affiliation{\institution{McGill University}}
  \email{bre@ndan.co}

  \maketitle

  \section{Introduction}

  Catholics have long contributed to the study of computer science, dating back to at least the early 6\textsuperscript{th} century. The \textit{computus} was originally developed to calculate the date of Easter in relation to Passover and the vernal equinox.

  In the 16th century, early Catholic astronomers such as Copernicus and Galileo used computational methods to predict the motion of celestial bodies, and developed telescopes to more carefully observe their orbits. Modern computers have been likened to telescopes in helping us to peer into the cosmos and empirically study natural laws. Just as more powerful telescopes revealed inconvenient truths for geocentrism, computers have sharpened man's understanding of the materiality of cognition and unveiled some of the mysteries of anthropological exceptionalism.

  Like AI, the heliocentric model was initially considered a challenge to the prevailing geocentric view of the universe and helped us to realize deeper appreciation for the enormity of God's creation and our special place within it as the legatees to Christ's mission on Earth. Yet, a tension remained between the triumph of humanity in the face of inhospitable odds, and the humility of our precaious position in a vast and apparently indifferent universe. Today, we are faced with a similar conundrum as we grapple with the implications of thinking machinery and our newfound ability to recreate many aspects the human mind in computational form.

  To deal with the dawning materialism of the modern age, Ren\'e Descartes, a Catholic philosopher, developed a theory of dualism that contends the mind and body were separate entities. This view has influenced the two modern schools of neural and symbolist AI, which imagines a similar duality between the material and platonic forms of intelligence. Recent work attempts to unify neurosymbolism, which seeks to bridge the gap between these two perspectives.

  Nearly a century later, Blaise Pascal, a 17\textsuperscript{th} century Catholic mathematician, developed one of the first mechanical calculators, the Pascaline, which was used to perform simple arithmetic operations. Pascal also developed a theory of probability which has been influential in the development of machine learning.

  Ludwig Wittgenstein, a 20\textsuperscript{th} century Catholic philosopher and pioneer of ordinary language philosophy, introduced a theory of language games which continues to mystify and inspire AI experts. His work has been cited as a precusor to modern theories about the vector space model of language semantics, which underpins the modern theory large langauge models.

  Georges Lema\^itre, a Belgian priest and physicist, developed the Big Bang theory in 1932. This theory was influential in the development of modern cosmology. He was an early adopter of computers to model the interaction between charged particles and Earth's magnetic field among other cosmological phenomena and was instrumental in early CS education in Belgium.

  More recently, the first Ph.D. in computer science was awarded in 1965 to a Catholic nun, Sister Mary Keller, who studied automatic differentiation and inductive inference. These topics are central in modern machine learning and continue to fascinate researchers in programming language theory and statistical inference.

  In this paper, we survey Catholic contributions and attitudes towards the subject of computation. We argue for a Catholic approach to AI, which is grounded in the values of universal accessibility, rooted in faith and reason and informed by the latest developments in computer science.

 \section{Gnostics}

  Despite its centuries-long history, computer science is still a nascent discipline, and there are many things we do not know about the origin of life, the nature of consciousness, and the limits of computation. The Catholic church has a long history of engaging with these questions from its beginning, and has produced a number of philosophical and theological treaties on the subject.

  Early Gnostic gospels describe the creation of the world as a kind of simulation, and the human soul as a kind of divine spark which has been imprisoned in a fallen material world, and through obtaining Gnosis one is able to escape from it and return to the divine realm. These teachings were condemned as heretical by the early Church, but have been influential in the broader culture, featuring prominently in the popular science fiction series, \textit{The Matrix}, which posits a simulated reality inhabited by stranded human souls who must learn secret knowledge to defeat sinister agents and free themselves from it. Far from being a fringe belief, these ideas are deeply ingrained in the hacker culture and embedded in the popular imagination.

  These views are echoed by Cartesian dualism, simulationism and the modern transhumanist movement, which seeks to transcend the limitations of the human body and mind through the use of technology. In contrast, the Catholic church teaches that the human body is a gift from God, and we should not seek to escape from it, but rather seek to obtain a more wholesome union between the body and soul.

  Like the Gnostics, modern CS has a reputation of being somewhat esoteric, and many of its practitioners fashion themselves after a kind of secular clergy, featuring cults of personality and benevolent dictators for life (BDFLs). The purpose of CS is not to create a new religion for the cognoscenti, but to help people learn, communicate and coordinate. Technology must be used to unify, and not to obtain secret knowledge for oneself and the privileged few. We should be cautious of the dangers of technocracy, and the occult ideologies that often accompany it.

  \section{Apologetics}

  The Catholic Church has a long history of engaging with the sciences, and has produced a number of apologetics defending the unity of faith and reason. It has long held that faith and reason are not in conflict, but are complementary and reciprocal ways of understanding God's creation. The Church teaches us the human mind is a gift from God, and that it is our duty to use it to the best of our ability.

  Statistical techniques appeal to faith when they are used to make predictions about the future. If a statistical model is faithful to the data, we can have faith in its predictions. But as Thomas Aquinas argues, faith is not blind, it is based on reason. We can reason about the consequences of statistical models, and use them to make better decisions. Formally modeling statistical techniques in a logical framework allows us to rigorously study their faithfulness. This is what probabilistic programming is all about.

  \section{Opportunities for dialogue}

  We will now discuss some opportunities for dialogue between the Catholic Church and the field of computer science. We will focus on three areas: the limits of computation, the relation between the mind and body, and the distinction between normative and descriptive statements.

  \subsection{TCS and the limits of computation}

  Under the banner of rational transhumanism, technological cults have emerged pushing the view of digital superintelligence and panpsychism. This is not only incompatible with a rationalist understanding, but anathematic to the Catholic faith.

  As Pope Francis argues, limits are ``frequently overlooked in our current technocratic and efficiency-oriented mentality'' and yet ``decisive for personal and social development''. This parallels the theory of computation, which studies the boundaries of what can and cannot be computed and how much power it takes to do. TCS offers a principled way to reason about the limits of unbridled computationalism, and computer science should ignore these limits at their own peril.

  As argued by Roger Penrose, the ability of the mind to see truth cannot be computed on a Turing machine nor reduced to blind calculation.

  \subsection{Ensouled bodies and embodied agents}

  Properly viewed, thinking machines are not a threat to human dignity, but an opportunity to better understand the nature of intelligence and its relation to the soul. The development of thinking machines can be seen as a kind of extension of the mind, just as the body is an extension of the soul. Like a mechanical calculator, we can use it to understand and better appreciate the structure of the universe.

  \subsection{Modal logic and reachable worlds}

  The entanglement between ``is'' and ``ought'', normative and descriptive statements is a central theme in the philosophy of mathematics. What is true by fiat is not necessarily true by necessity. Computation is a constructive kind of mathematics, and shares many similarities with Catholic outreach to the poor and marginalized. We do not impose our beliefs on others, but offer a moral way to understand the world and our place within it.

  Modal logic is an interesting move here. Type theory says, ``here is the way things are''. This is the ivory tower. Modal logic says, ``here is the way things could or should be''. This allows us to encode the distinction between normative and descriptive statements, and to reason about the consequences of certain actions. It is the language of possibility, and offers a way to encode intent and possibility in a formal language.

  \bibliography{bib}
\end{document}