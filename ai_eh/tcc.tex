\pdfoutput=1
%% For double-blind review submission, w/o CCS and ACM Reference (max submission space)
%\documentclass[sigplan,10pt,review,anonymous]{acmart}
%\settopmatter{printfolios=false,printccs=false,printacmref=false}
%% For double-blind review submission, w/ CCS and ACM Reference
%\documentclass[sigplan,review,anonymous]{acmart}\settopmatter{printfolios=true}
%% For single-blind review submission, w/o CCS and ACM Reference (max submission space)
%\documentclass[sigplan,review]{acmart}\settopmatter{printfolios=true,printccs=false,printacmref=false}
%% For single-blind review submission, w/ CCS and ACM Reference
%\documentclass[sigplan,review]{acmart}\settopmatter{printfolios=true}
%% For final camera-ready submission, w/ required CCS and ACM Reference
\documentclass[sigplan,nonacm]{acmart}\settopmatter{printfolios=false,printccs=false,printacmref=false}

%% Bibliography style
\bibliographystyle{acmart}

\begin{document}
  \title{Catholic Viewpoints on Thinking Machinery}
  \begin{abstract}
  The discovery of a mechanistic means of simulating mental processes raises a variety of philosophical and theological questions. In this paper, we survey Catholic contributions towards automation, arguing the coexistience of thinking machines is not only compatible with the Catholic faith, but offers a more nuanced understanding of the relation between spirit and flesh. Finally, we survey historical and contemporary Catholic perspectives on thinking machinery, and retrace current trends in computer science to their theological roots, reflecting on how the Catholic intellectual tradition has shaped the development of artificial intelligence.
  \end{abstract}

  \author{Breandan Considine}
  \affiliation{\institution{McGill University}}
  \email{bre@ndan.co}

  \maketitle

  \section{Introduction}

  Catholics have long contributed to the study of computer science, dating back as far as the early 6\textsuperscript{th} century. The \textit{computus} was originally developed to calculate the date of Easter in relation to Passover and the vernal equinox.

  In the 16th century, early Catholic astronomers such as \textbf{Copernicus} and \textbf{Galileo} used computational methods to predict the motion of celestial bodies, and developed telescopes to more carefully observe their orbits. Modern computers have been likened to telescopes in helping us to peer at distant truths and empirically study natural laws. As more powerful telescopes began to displace the prevailing geocentric perspective, so too have computers sharpened man's understanding of cognitive processes and unveiled some of the mysteries of anthropological exceptionalism.

  The paradigm shift ushered in by the telescope helped man to realize a deeper appreciation for the enormity of God's creation and our special place within it as the legatees to Christ's mission. Yet, an antimony lingered between the triumph of humanity against cosmic odds, and our parochial self-importance amidst a vast and indifferent universe. Today, we are faced with a similar conundrum as we grapple with the implications of thinking machinery and our ability to recreate aspects of the human mind in computational form.

  In anticipation of the dawning physicalism of the modern age, \textbf{Ren\'e Descartes}, a Catholic philosopher, developed a dual theory of mind and matter as separate substances~\cite{descartes1641meditationes}. Decartes' analytic geometry inspired both Leibniz and Newton, whose later work on differential calculus laid the foundation for gradient descent, which powers the modern wave of deep learning. Today, the echoes of Cartesianism can be heard in the two schools of neural and symbolic AI, as they contemplate a similar duality between the geometric and algebraic, materialist and idealist forms of intelligence.

  At the heart of Cartesian theories was the recognition that physical processes not only obeyed rational laws, but mental processes could be physically engineered. So began a centuries-long project to realize those ambitions by enlisting physics to do our thinking, culminating in devices that modern programmers would begin to call computers and AI.

  A Catholic mathematician and contemporary of Descartes, \textbf{Blaise Pascal}, developed one of the earliest calculators, the Pascaline, to automate taxes. An avid theologian, the basis for Pascal's beliefs are argued in his \textit{Thoughts}~\cite{pascal1670pensees} wherein his \textit{Discourse on the Machine} and famous wager on God's existence can be found. The theory of probability he developed was highly influential on modern AI, and the popular programming langauge and a semiconductor architecture powering the first wave of deep learning now bears his name.

  Essential to the computationalist project was a mechanism for communicating intent. \textbf{Ludwig Wittgenstein}, a 20\textsuperscript{th} century Catholic philosopher and pioneer of ordinary language philosophy~\cite{wittgenstein1929some}. Like Pascal, Wittgenstein was a fideist and worldly academic who took a keen interest in philosophy and theology, first proposing the idea of language as a game to convey intent. His work has been cited as a precusor to the game-theoretic approach to formal language semantics and the vector space model of language, which informs our modern understanding of large langauge models.

  \textbf{Fr. Georges Lema\^itre}, a Belgian priest and physicist, first proposed the Big Bang~\cite{lemaitre1931beginning}, a theory of cosmological expansion in 1932, later observationally supported. Lema\^itre was an early adopter of computer modeling to study the interaction between charged particles and Earth's magnetosphere and was instrumental in early computing education in Belgium. Like many of his Catholic predecessors, Lema\^itre held that faith and reason were independent, but never in conflict.

  More recently, the first Ph.D. in computer science was awarded in 1965 to a Catholic nun, \textbf{Sr. Mary Keller}, who wrote her doctoral thesis~\cite{keller1965inductive} on automatic differentiation (AD) and inductive inference, topics of central importance that continue to occupy researchers in programming language and statistics today. AD would later prove instrumental in the development and popularization of deep learning.

  In this paper, we survey two millennia of Catholic contributions and perspectives on the subject of computation. We argue for a Catholic approach to the pursuit of artificial intelligence, which is grounded in the values of universal accessibility, rooted in faith and reason and informed by the latest developments in mathematics and computer science, in particular, formal methods, probability and humanities.

 \section{Gnostics}

  Despite its roots in antiquity, computer science is still a nascent discipline, and there are many things we do not yet know about the origin of life, the nature of consciousness, and the limits of computation. The Catholic Church has a long history of engaging with these questions starting from its inception, and has contributed a variety of theological and philosophical ideas foreshadowing modern AI.

  Early Gnostic gospels dating back to the first century describe the creation of the world as a kind of simulation, and the human soul as a kind of divine spark imprisoned in a fallen material world. While considered heretical by the early Church, these teachings have been influential in the broader culture, featuring prominently in the cult sci-fi series, \textit{The Matrix}, which posits a simulated reality inhabited by stranded human souls who must learn secret knowledge to vanquish sinister agents and free themselves from it. Far from being a fringe belief, these ideas are deeply ingrained in the hacker ethos and embedded in the popular imagination.

  Like the Gnostics, modern computer science has a reputation for being fairly esoteric, and many of its practitioners fashion themselves after a kind of secular clergy, featuring technical evangelists, cults of personality, and self-professed prophetic figures claiming to possess eschatological knowledge about the future of humanity. This pseudo-spirituality is not accidental, but reflects a pastiche of religious symbolism that has been appropriated and repurposed by the tech industry to lend itself an air of mystique and credibility.

  Such beliefs have become especially prevalent in AI, manifesting as accelerationism and the transhumanist movement, which seeks to transcend the human form and achieve a kind of digital immortality by uploading one's mind into cyberspace. Perhaps unwittingly, these voices often echo Gnostic themes, such as a dualistic cosmology, rejection of the material world, pursuit of secret knowledge, and the desire to escape the earthly realm into a higher existence.

  In contrast, the contemporary Catholic Church teaches the body is a gift, that we should not seek to escape it, but rather unite body and soul in praising God, professing the faith, and helping the needy. In a similar spirit, the purpose of computer science should be oriented towards studying natural law, teaching programming, and helping the technologically challenged. It would be wise to be wary of the dangers of technocracy, and the occult ideologies that often arise in pursuing technological progress for its own sake.

  Properly oriented, technology should not be to used to hoard knowledge for personal salvation, but to serve the greater good and intrinsic human dignity. Here, the church leads by example, offering a variety of paths for the faithful, from founding the first universities, hospitals and orphanages, based on mutual respect for faith and reason, guided by spiritual discernment in accordance with the principles of natural law, and discharged through the works of mercy.

  \section{Scholastics}

  \textbf{St. Anslem of Canterbury} first committed the ontological argument to writing in his seminal text, the \textit{Proslogion}, the prototypic example of a logical proof for the existence of God~\cite{anslem1078fides}. This argument starts with the premise that God is the greatest conceivable being, and even unbelievers must concede that such a being must exist in the mind. If such a being exists in the mind, then it must also exist in reality, because any being which exists in both reality and the mind would be even greater than one who exists solely in the mind. Therefor, we conclude that God must too exist in reality.

  This argument inspired a number of saints and scholars, including most recently, the famous mathematician, Kurt G\"odel. Although himself not a Catholic, G\"odel was a self-professed theist and lecturer at the Catholic University of Notre Dame in South Bend, Indiana. In addition to his landmark incompleteness theorems in mathematical logic, G\"odel also developed a modal ontological argument~\cite{wang1997logical}, adapted from Descartes and Leibniz. Modal logic is a kind of logic developed to help reason about necessity and possibility, and a well-studied alternative to classical logic in computer science to help us reason about the behavior of programs.

  \textbf{St. Thomas Aquinas}, a Dominican friar and priest, who developed five proofs for the existence of God, including another modal argument known as the \textit{Third Way}~\cite{aquinas2008summa}. This proof rests on the idea that everything is either contingent or necessary. If we assume the natural world is contingent, all contingent things must have a necessary cause, or else they would not exist. Therefor, Aquinas concluded, God must be that cause. Since the natural world was created by God, Aquinas believed it was possible to become closer to Him by studying nature, forming the basis for the scientific method.

  St. Aquinas held the belief that animals had no moral rights or responsibilities, and thus would have likely had even less sympathy for thinking machines as moral patients. However, Aquinas also believed that we should treat animals kindly and not cause them undue suffering, to avoid inuring ourselves to cruelty and to cultivate a sense of compassion and empathy. Likewise, even though thinking machines have no moral rights, we ought not treat them unkindly, to avoid becoming hard-hearted towards our fellow man.

  \textbf{Fr. Duns Scotus} was a Fransiscan friar and apologist who pondered the nature of universals (e.g., color and shape) and arrived at the conclusion that universal forms exist in reality, a position now known as scholastic realism. His work had a significant influence on Charles Saunders Pierce -- the founder of pragmatism -- and Clarence Irving Lewis -- the founder of modal logic -- which are both widely used in computer science today. These two subjects have been instrumental in the development of formal methods, such as model checking and theorem proving, which are used extensively in computer science to verify the correctness of software and hardware systems. In a roundabout way, CS can be seen as realizing a kind of scholastic realism, by instantiating abstract mathematical models in physical hardware.

  \textbf{Fr. William of Occam} was another 14\textsuperscript{th} century Franciscan and widely regarded as one of the greatest logicians of the Middle Ages. A student at Oxford University, he later developed the now famous principle of Occam's razor, which posits that the simplest explanation is usually the best one~\cite{occam1495quaestiones}. This principle, also known as \textit{parsimony}, has been widely influential in constraint satisfaction and natural language processing as a heuristic for selecting the best hypothesis among a set of possible alternatives. Occam opposed Scotus, and believed that universal forms exist only in the mind and have no external reality, a position which came to be known as nominalism. Nominalism is now amidst a revival in the field of programming language theory under the banner of nominal sets and automata, used in probabilistic programming and certain type theories. As we know from Sacred Scripture, God calls us each by name (Isaiah 43:1-7), however, whether God calls his other creations by name, or if names are simply a human convention, is still debated.

  \section{Ethics}

  In his 2024 World Peace Day message,~\footnote{\url{https://www.vatican.va/content/francesco/en/messages/peace/documents/20231208-messaggio-57giornatamondiale-pace2024.html}} \textbf{Pope Francis} highlighted the importance of the ethical development of AI, drawing special attention to the dangers of weaponization, surveillance, and the risk of bias and discrimination in AI systems. He calls for a more human-centric AI approach, which respects inherent human dignity, and the importance of limits in the development of AI systems, and positive best-practices, including transparency, security and reliability.

  This advice is in keeping with centuries of teaching on the importance of ethics in technology, tracing back to Sacred Scripture, which teaches us the Holy Spirit is the source of all wisdom and insight, providing man with ``wisdom, understanding, knowledge and all kinds of skills'' (Exodus 31:3). Yet, scripture also teaches us that wisdom and knowledge are very different, and that knowledge without wisdom can lead to ruin, as we are reminded of in the Tower of Babel (Genesis 11:1-9) and the fall of Sodom and Gomorrah (Genesis 18:16-19:29). These calamities serve as poignant reminders of the dangers of hubris and the importance of humility and God-fearing reverence in the pursuit of knowledge.

  Sloth, depravity, moral turpitude, vigilantism, indecency, willful ignorance and incompetence, subversion of the truth, libel, identity theft, swindling, malfeasance and conartistry -- the internet is rife with examples of unethical behavior that AI can exacerbate. A litany of finger-wagging is devoted to negative ethics and technological misuse, which are widely discussed in the literature and too often taken as inspirational material instead of the cautionary advice it is intended to serve. Instead, we will focus primarily on positive ethics.

  A positive ethics is one that is grounded in the pursuit of truth, goodness and beauty, and the belief that knowledge springs forth from God, but we are stewards of knowledge, not its masters. Many of the greatest scientists and mathematicians in history never lost sight of this fact, and saw themselves as seekers of truth, not creators of it; they saw their work as a form of worship, were guided by a deep sense of humility and reverence for the natural world, and sought to understand the mind of God by studying the beauty of His creation. Charity, thus understood, is pursuing and open-sourcing technology for the sake of improving the human condition, and not for recognition or personal salvation.

  In Catholicism, charity or mercy is a very practical matter, exemplified by the corporal works, which emphasize feeding the hungry, sheltering the homeless, ransoming captives, and burying the dead, as well as the spiritual works, which include instructing the ignorant, counseling the doubtful, and admonishing sinners. Christ calls us each to do these things, and teaches that whatever we do for the least of our brothers, we do unto Him (Matthew 25:40).

  A central tenet of Catholic ethics is the infinite dignity of every human being, and the importance of treating our fellow man with respect and compassion. This is reflected in the church's teachings on the sanctity of life, the importance of the family, and the dignity of work, causes to which the church has long championed, having founded many of the earliest hospitals and universities. As techologists, we should strive to emulate these values, and ensure that AI is used to promote human dignity and heathly human relationships.

  To honor the dignity of our fellow man, God has instructed us to ``not bear false witness against thy neighbor'' (Exodus 20:16). Not only are the faithful explicitly forbidden from practicing wilful deception, Sacred Scripture further prohibits the use of dishonest physical instruments, ``Do not use dishonest standards when measuring length, weight or quantity. Use honest scales and honest weights, an honest ephah and an honest hin.'' (Leviticus 19:35-36)

  From the standpoint of jurisprudence and religious ethics, AI is essentially a more sophisticated scale whose use and misuse can have indirect effects on society. Yet, the virtue of honesty is easily overlooked in the tech industry, which often prioritizes speed and efficiency over accuracy and truthfulness. This applies to both institutional honesty, and personal accountability. In a positive ethics of AI, developers should strive to be honest in our intentions, transparent about the limitations of our models, and forthright about the consequences of misuse. From a technical standpoint, we are encouraged to use and share open source code, anonymized data and model parameters -- the gold standard for transparency and accountability in computer science.

  Open source serves as both a form of charity and a form of honesty. It is a way to give back to the community, to share knowledge and to help others build things. It also provides a way to check that a program is behaving as intended. Yet, we must be careful to share only what is necessary, and respect the privacy and intellectual property rights of those who would rather remain anonymous. Likewise, computer scientists must remain conscious of the second-order effects, such as enabling bad actors to achieve nefarious goals.

  \section{Opportunities for dialogue}

  We will now discuss some opportunities for dialogue between the Catholic Church and the field of computer science. We will focus on three areas: the limits of computation, the relation between the mind and body, and the distinction between normative and descriptive statements.

  \subsection{The limits of computation}

  Under the banner of rational transhumanism, technofuturist cults have emerged pushing the view of digital superintelligence and panpsychism. This is not only incompatible with a rationalist understanding, but anathematic to the Catholic faith. As Pope Francis argues, limits are ``frequently overlooked in our current technocratic and efficiency-oriented mentality'' and yet ``decisive for personal and social development''. This parallels the theory of computation, which studies the boundaries of what can and cannot be computed and how much power it will take to accomplish. Theoretical computer science offers a principled way to reason about the limits of unbridled computationalism, and computer science should ignore these limits at their own peril.

%  As argued by Roger Penrose, the ability of the mind to see truth cannot be computed on a Turing machine nor reduced to blind calculation.

  \subsection{Grounded theory and symbolic interactionism}

  Anslem Strauss and Barney Glaser founded grounded theory in the 1960s, which became an influential approach in the social sciences. Their basic idea was to study social phenomena in a systematic way, starting with data and developing a theory from it, rather than starting with theory and looking for data to support it. Strauss and Glaser were heavily influenced by the American school of pragmatism through George Mead's work on symbolic interactionism, a theory of social behavior that emphasizes the importance of symbols and language in shaping our understanding of the world. Grounded theory emphasizes \textit{coding}, and has been influential in the development of qualitative research methods in computer science, such as natural language processing and sentiment analysis in social computing. Today, these tools are being used to analyze the social coding practices of open source communities, and to develop new ways of understanding the social dynamics of online communities.


  \subsection{Modal logic and reachable worlds}

  The entanglement between ``is'' and ``ought'', normative and descriptive statements is a central theme in the philosophy of mathematics. What is true by fiat is not necessarily true by necessity. Computation is a constructive kind of mathematics, and shares many similarities with Catholic outreach to the poor and marginalized. We do not impose our beliefs on others, but offer a moral way to understand the world and our place within it and reach out to those in need.

  Clarence Irving Lewis, the founder of modal logic, argued that the distinction between necessity and possibility is fundamental to understanding the nature of reality. He started with Russell's Principia Mathematica, but took exception to the conflation of material and strict implication. This break from the logical positivists and the subsequent development of modal logic was influential in the development of formal methods in computer science, such as model checking and theorem proving. He also coined the term ``qualia'' to describe the subjective experience of consciousness, which widely adopted in cognitive science and AI.

  Modal logic is an interesting move here. Type theory says, ``here is the way things are''. Modal logic says, ``here is the way things could or ought to be''. This allows us to encode the distinction between normative and descriptive statements, and to reason about the consequences of certain actions. It offers a way to distinguish intent from possibility in a formal language and (seems to be?) the ambient context or a natural extension of automata theory and formal languages.

  \subsection{Probabilistic programming}

  Statistical techniques appeal to faith when they are used to make predictions about the future. If a statistical model is faithful to the data, we can have faith in its predictions. But as Thomas Aquinas argues, faith is not blind, but rests upon reason and as we know from the Bible, faith without deeds is dead (James 2:14-26). In order to take interventions with confidence, probabilistic models must be grounded in a logical framework, and yield testable predictions. Formally modeling statistical techniques in a logical framework allows us to rigorously study their faithfulness. This is what probabilistic programming is all about, i.e., building an executable model of a stochastic system, simulating trajectories and reasoning about the consequences of intervention.

  \subsection{Ensouled bodies and embodied agents}

  Properly viewed, thinking machines are not a threat to human dignity, but an opportunity to better understand the nature of intelligence and its relation to the soul. The development of thinking machines can be seen as a kind of extension of the mind, just as the body is an extension of the soul. Like a mechanical calculator, we can use it to understand and better appreciate the structure of the universe. Likewise, physically embodied agents can be used to reduce the burden of manual labor, such as transporting goods and humans, planting crops and manufacturing things, freeing mankind to leave the fields, factories and mines and follow more ennobling career paths such as spiritual, intellectual and humanistic pursuits.

  \clearpage
  \section{Conclusion}

  The development of AI is but a small step in a centuries-long project to understand the human mind. Since the dawn of the anthropocene, man has sought to understand and harness nature to reduce suffering and improve the human condition. Catholicism has been a steadfast ally in this endeavor, providing a moral compass and a rich intellectual tradition to guide us in our pursuit of knowledge and wisdom. As we continue to develop AI, we should seek to emulate its values of mercy, humility, and reverence for the natural world. By doing so, we can ensure that AI is used to promote human dignity and the greater good, and not to further the interests of a select few.

  \bibliography{bib}
\clearpage
\end{document}