\pdfoutput=1
%% For double-blind review submission, w/o CCS and ACM Reference (max submission space)
%\documentclass[sigplan,10pt,review,anonymous]{acmart}
%\settopmatter{printfolios=false,printccs=false,printacmref=false}
%% For double-blind review submission, w/ CCS and ACM Reference
%\documentclass[sigplan,review,anonymous]{acmart}\settopmatter{printfolios=true}
%% For single-blind review submission, w/o CCS and ACM Reference (max submission space)
%\documentclass[sigplan,review]{acmart}\settopmatter{printfolios=true,printccs=false,printacmref=false}
%% For single-blind review submission, w/ CCS and ACM Reference
%\documentclass[sigplan,review]{acmart}\settopmatter{printfolios=true}
%% For final camera-ready submission, w/ required CCS and ACM Reference
\documentclass[sigplan,nonacm]{acmart}\settopmatter{printfolios=false,printccs=false,printacmref=false}

%% Bibliography style
\bibliographystyle{acmart}

\begin{document}
  \title{Catholic Viewpoints on Thinking Machinery}
  \begin{abstract}
  The discovery of an efficient means of simulating mental processes raises a variety of philosophical and theological questions. In this paper, we survey historical Catholic contributions and contemporary attitudes towards intellectual automation. In constrast with predominant secular views, we argue the development of thinking machines is not only compatible with the Catholic faith, but offers a more nuanced understanding of the relation between spirit and flesh. We survey historical and contemporary Catholic perspectives on thinking machinery, and how they inform our understanding of intelligence and the limits of computation.
  \end{abstract}

  \author{Breandan Considine}
  \affiliation{\institution{McGill University}}
  \email{bre@ndan.co}

  \maketitle

  \section{Introduction}

  Catholics have long contributed to the study of computer science, dating back to at least the early 6\textsuperscript{th} century. The \textit{computus} was originally developed to calculate the date of Easter in relation to Passover and the vernal equinox.

  In the 16th century, early Catholic astronomers such as \textbf{Copernicus} and \textbf{Galileo} used computational methods to predict the motion of celestial bodies, and developed telescopes to more carefully observe their orbits. Modern computers have been likened to telescopes in helping us to peer at distant truths and empirically study natural laws. As more powerful telescopes began to displace the prevailing geocentric perspective, so too have computers sharpened man's understanding of cognitive processes and unveiled some of the mysteries of anthropological exceptionalism.

  Initially considered a challenge to geocentrism, the telescope and paradigm shift it ushered helped man to realize a deeper appreciation for the enormity of God's creation and our special place within it as the legatees to Christ's mission on Earth. Yet, an antimony remained between the triumph of humanity against cosmic odds, and our parochial self-importance amidst a vast and indifferent universe. Today, we are faced with a similar conundrum as we grapple with the implications of thinking machinery and our ability to recreate aspects the human mind in computational form.

  In anticipation of the dawning physicalism of the modern age, \textbf{Ren\'e Descartes}, a Catholic philosopher, developed a dual theory of mind and matter as separate substances. Decartes' analytic geometry inspired both Leibniz and Newton, whose later work on differential calculus laid the foundation for gradient descent. Today, the echoes of Cartesianism can be heard in the two schools of neural and symbolic AI, which contemplate a similar duality between the geometric and algebraic, materialist and idealist forms of intelligence.

  At the heart of Cartesian theories was the recognition that physical processes not only obeyed rational laws, but mental processes could be physically engineered. So began a centuries-long project to realize those ambitions by enlisting physics to do our thinking, culminating in devices that modern programmers would begin to call computers and AI.

  A Catholic mathematician and contemporary of Descartes, \textbf{Blaise Pascal}, developed one of the earliest calculators, the Pascaline, to automate taxes. An avid theologian, the basis for Pascal's beliefs are argued in his \textit{Thoughts} and \textit{Discourse on the Machine} wherein his famous wager on God's existence can be found. The theory of probability he developed was highly influential on modern AI, and the popular programming langauge and a semiconductor architecture powering the first wave of deep learning now bears his name.

  Essential to the computationalist project was a mechanism for communicating intent. \textbf{Ludwig Wittgenstein}, a 20\textsuperscript{th} century Catholic philosopher and pioneer of ordinary language philosophy. Like Pascal, Wittgenstein was a fidest and worldly academic who took a keen interest in philosophy and theology, first proposing the idea of language as a game to convey intent. His work has been cited as a precusor to game-theoretic approach to formal language semantics and the vector space model of language, which informs our modern understanding of large langauge models.

  \textbf{Fr. Georges Lema\^itre}, a Belgian priest and physicist, first proposed the Big Bang, a theory of cosmological expansion that was later empirically tested and accepted, in 1932. Lema\^itre was an early adopter of computer technology to study the interaction between charged particles and Earth's magnetic field, among other cosmological phenomena, and was instrumental in early computing education in Belgium. Like many of his Catholic predecessors, Lema\^itre held that faith and reason were independent, but never in conflict.

  More recently, the first Ph.D. in computer science was awarded in 1965 to a Catholic nun, \textbf{Sr. Mary Keller}, who wrote her doctoral thesis on automatic differentiation (AD) and inductive inference, both topics of essential importance that continue to occupy researchers in programming language and statistics today. AD would later prove instrumental in the development and popularization of deep learning.

  In this paper, we survey two millennia of Catholic contributions and perspectives on the subject of computation. We argue for a Catholic approach to the pursuit of artificial intelligence, which is grounded in the values of universal accessibility, rooted in faith and reason and informed by the latest developments in mathematics and computer science.

 \section{Gnostics}

  Despite its roots in antiquity, computer science is still a nascent discipline, and there are many things we do not know about the origin of life, the nature of consciousness, and the limits of computation. The Catholic church has a long history of engaging with these questions starting from its inception, and has produced a number of theological and doctrinal treaties on the subject throughout the centuries.

  Early Gnostic gospels dating back to the first century describe the creation of the world as a kind of simulation, and the human soul as a kind of divine spark imprisoned in a fallen material world. These teachings were considered heretical by the early Church, but have been influential in the broader culture, featuring prominently in the cult sci-fi series, \textit{The Matrix}, which posits a simulated reality inhabited by stranded human souls who must learn secret knowledge to vanquish sinister agents and free themselves from it. Far from being a fringe belief, these ideas are deeply ingrained in the hacker culture and embedded in the popular imagination.

  Like the Gnostics, modern computer science has a reputation for being somewhat esoteric, and many of its practitioners fashion themselves after a kind of secular clergy, featuring technical evangelists, cults of personality, and self-professed prophetic figures claiming to have special knowledge about the nature of computation. This terminology is not accidental, but reflects a pastiche of religious symbolism that has been appropriated and repurposed by the tech industry to lend itself an air of mystery and credibility.

  These beliefs have become especially prevalent in the field of AI, manifesting as simulationism and the modern transhumanist movement, which seeks to transcend the human body and achieve a kind of digital immortality by uploading one's mind into cyberspace. Unwittingly, these voices often echo Gnostic themes, such as a dualistic cosmology, rejection of the material world, pursuit of secret knowledge, and the desire to escape the earthly realm into a higher existence.

  In contrast, the church teaches us the body is a gift, and we should not seek to escape from it, but rather put it to good use by uniting body and soul in the praise of God, professing the faith, and helping the needy. Similarly, the purpose of computer science is to study natural law, teach programming in an easily-accessible manner, and help the digitally illerate interact with computers and each other. We are taught to be wary of the dangers of technocracy, and the occult ideologies that often arise in pursuit of technological progress.

  Properly oriented, technology should not be to used to secrete knowledge for personal salvation, but to serve the greater good and intrinsic human dignity. Here, the church leads by example, offering a variety of paths for the faithful, from the founding of the first universities, to developing the first hospitals and orphanages in the modern era, based on mutual respect for faith and reason, guided by spiritual discernment and the principles of charity and natural law.

  \section{Apologetics}

  The Catholic Church has a long history of engaging with the sciences, and has produced a number of apologetics defending the unity of faith and reason. The Church teaches us the human mind is a gift from God, and that it is our duty to use it to the best of our ability.

  An early proponent of natural law was Thomas Aquinas. Aquinas argued that the natural world was created by God, and that it was possible to reason about the nature of God by studying the natural world. He held that faith and reason are not in conflict, but are complementary and reciprocal ways of understanding God's creation.

  Statistical techniques appeal to faith when they are used to make predictions about the future. If a statistical model is faithful to the data, we can have faith in its predictions. But as Thomas Aquinas argues, faith is not blind, it is based on reason. We can reason about the consequences of statistical models, and use them to make better decisions. Formally modeling statistical techniques in a logical framework allows us to rigorously study their faithfulness. This is what probabilistic programming is all about.

  \section{Opportunities for dialogue}

  We will now discuss some opportunities for dialogue between the Catholic Church and the field of computer science. We will focus on three areas: the limits of computation, the relation between the mind and body, and the distinction between normative and descriptive statements.

  \subsection{TCS and the limits of computation}

  Under the banner of rational transhumanism, technological cults have emerged pushing the view of digital superintelligence and panpsychism. This is not only incompatible with a rationalist understanding, but anathematic to the Catholic faith.

  As Pope Francis argues, limits are ``frequently overlooked in our current technocratic and efficiency-oriented mentality'' and yet ``decisive for personal and social development''. This parallels the theory of computation, which studies the boundaries of what can and cannot be computed and how much power it takes to do. TCS offers a principled way to reason about the limits of unbridled computationalism, and computer science should ignore these limits at their own peril.

  As argued by Roger Penrose, the ability of the mind to see truth cannot be computed on a Turing machine nor reduced to blind calculation.

  \subsection{Ensouled bodies and embodied agents}

  Properly viewed, thinking machines are not a threat to human dignity, but an opportunity to better understand the nature of intelligence and its relation to the soul. The development of thinking machines can be seen as a kind of extension of the mind, just as the body is an extension of the soul. Like a mechanical calculator, we can use it to understand and better appreciate the structure of the universe.

  \subsection{Modal logic and reachable worlds}

  The entanglement between ``is'' and ``ought'', normative and descriptive statements is a central theme in the philosophy of mathematics. What is true by fiat is not necessarily true by necessity. Computation is a constructive kind of mathematics, and shares many similarities with Catholic outreach to the poor and marginalized. We do not impose our beliefs on others, but offer a moral way to understand the world and our place within it.

  Modal logic is an interesting move here. Type theory says, ``here is the way things are''. This is the ivory tower. Modal logic says, ``here is the way things could or should be''. This allows us to encode the distinction between normative and descriptive statements, and to reason about the consequences of certain actions. It is the language of possibility, and offers a way to encode intent and possibility in a formal language.

  \bibliography{bib}
\end{document}