\pdfoutput=1
%% For double-blind review submission, w/o CCS and ACM Reference (max submission space)
%\documentclass[sigplan,10pt,review,anonymous]{acmart}
%\settopmatter{printfolios=false,printccs=false,printacmref=false}
%% For double-blind review submission, w/ CCS and ACM Reference
%\documentclass[sigplan,review,anonymous]{acmart}\settopmatter{printfolios=true}
%% For single-blind review submission, w/o CCS and ACM Reference (max submission space)
%\documentclass[sigplan,review]{acmart}\settopmatter{printfolios=true,printccs=false,printacmref=false}
%% For single-blind review submission, w/ CCS and ACM Reference
%\documentclass[sigplan,review]{acmart}\settopmatter{printfolios=true}
%% For final camera-ready submission, w/ required CCS and ACM Reference
\documentclass[sigplan,nonacm]{acmart}\settopmatter{printfolios=false,printccs=false,printacmref=false}

%% Bibliography style
\bibliographystyle{acmart}

\begin{document}
  \title{Catholic Viewpoints on Thinking Machines}
  \begin{abstract}
    The emergence of an efficient means of simulating human mental activity has raised a number of philosophical and theological questions. In this paper, we survey historical Catholic contributions and attitudes towards intellectual automation and how the development of thinking machines can be reconciled with the Catholic faith. In constrast with predominant secular views, we argue that the development of thinking machines is not only compatible with Catholic doctrine, but offers a more nuanced understanding of the relation between mind, body and soul. We apply grounded theory to analyze the historical and contemporary Catholic perspectives on the subject, and how they inform our understanding of the nature of intelligence and the limits of computation.
  \end{abstract}

  \author{Breandan Considine}
  \affiliation{\institution{McGill University}}
  \email{bre@ndan.co}

  \maketitle

  \section{Introduction}

  Catholics have long contributed to the study of computer science, dating back to at least the early 6\textsuperscript{th} century. The \textit{computus} was originally developed to calculate the date of Easter in relation to Passover and the vernal equinox.

  In the 16th century, early Catholic astronomers such as Copernicus and Galileo used computational methods to calculate the motion of celestial bodies, and developed telescopes to more carefully observe them. Modern computers can be seen as a continuation of this tradition, in that they can be used to bring the distant future or past into greater focus and study the interplay between natural laws in greater detail.

  Around the same time, Ren\'e Descartes, a Catholic philosopher, developed a theory of dualism which has been influential in the development of AI. Descartes contended the mind and body were separate entities, and that the mind could be understood as a kind of machine. This view has been influential in the development of neurosymbolic AI, which imagines a similar disparity between the material and spiritual forms of intelligence.

  Blaise Pascal, a 17\textsuperscript{th} century Catholic mathematician, developed one of the first mechanical calculators, the Pascaline, which was used to perform arithmetic operations. Pascal also developed a theory of probability which has been influential in the development of machine learning.

  Ludwig Wittgenstein, an early 20\textsuperscript{th} century Catholic philosopher, developed a philosophy of language games which continues to mystify AI experts. His work has been cited as a precusor to modern theories about the vector space model of language semantics, which underpins many large langauge models.

  The first Ph.D. in computer science was awarded in 1965 to a Catholic nun, Sister Mary Keller, who studied automatic differentiation and inductive inference, central techniques used in modern machine learning and topics which continue to fascinate programming language theory and statistical inference.

  In this paper, we survey Catholic contributions and attitudes towards the subject of computation, and how it informs our understanding of the mind, body and soul.

 \section{Apologetics}

  Despite its centuries-long history, computer science is still a nascent discipline, and there are many things we do not know about the origin of life, the nature of consciousness, and the limits of computation. The Catholic Church has a long history of engaging with these questions, and has produced a number of apologetics on the subject.

  Early Christian texts such as the \textit{Apocryphon of John} describe the creation of the world as a kind of simulation, and the human soul as a kind of divine spark which has been trapped in the material world. This view is echoed in the work of St. Augustine, who described the soul as a kind of mirror which reflects the divine light.

  Like the Gnostics, computer science has a reputation of being somewhat esoteric, and its practitioners are often seen as a kind of secular priesthood. Howver, the purpose of computer science is not to create a new religion, but to help people learn and coordinate. Technology should be used to unify, and not to exploit or manipulate others. We should be cautious of the dangers of technocracy, and the occult ideologies that often accompany it.

  Later texts such as the \textit{Summa Theologica}

  \section{Theory of computation and the limits of the mind}

  Under the banner of rational transhumanism, technological cults have emerged pushing the view of digital superintelligence and panpsychism. This is not only incompatible with a rationalist understanding, but anathematic to the Catholic faith. As argued by Roger Penrose, the ability of the mind to see truth cannot be computed on a Turing machine nor reduced to blind calculation.

  As Pope Francis argues, limits are ``frequently overlooked in our current technocratic and efficiency-oriented mentality'' and yet ``decisive for personal and social development''. This parallels the theory of computation, which studies the boundaries of what can and cannot be computed and how much power it takes to do. TCS offers a principled way to reason about the limits of unbridled computationalism, and computer science should ignore these limits at their own peril.

  \section{Ensouled beings and embodied intelligence}

  Properly viewed, thinking machines are not a threat to human dignity, but an opportunity to better understand the nature of intelligence and its relation to the soul. The development of thinking machines can be seen as a kind of extension of the body, just as the body is an extension of the soul. Like a mechanical calculator we can use to understand and better appreciate the fine structure of the universe.

  \section{Modal logic and outreach to accessible worlds}

  The entanglement between "is" and "ought", normative and descriptive statements is a central theme in the philosophy of mathematics. What is true by fiat is not necessarily true by necessity. Computation is a constructive kind of mathematics, and shares many similarities with Catholic outreach to the poor and marginalized. We do not impose our beliefs on others, but offer a moral way to understand the world and our place in it.

  Modal logic is an interesting move here. Type theory says, "here is the way things are". This is the ivory tower. Modal logic says, "here is the way things could or should be". This allows us to encode the distinction between normative and descriptive statements, and to reason about the consequences of certain actions. It is the language of possibility, and offers a way to translate between formal and natural languages.

  \section{Faith and reason}

  Statistical techniques appeal to faith when they are used to make predictions about the future. If a statistical model is faithful to the data, we can have faith in its predictions. But as Thomas Aquinas argues, faith is not blind, it is based on reason. We can reason about the consequences of statistical models, and use them to make better decisions. Formally modeling statistical techniques in a logical framework allows us to rigorously study their faithfulness. This is what probabilistic programming is all about.

  \bibliography{bib}
\end{document}