\pdfoutput=1
%% For double-blind review submission, w/o CCS and ACM Reference (max submission space)
%\documentclass[sigplan,10pt,review,anonymous]{acmart}
%\settopmatter{printfolios=false,printccs=false,printacmref=false}
%% For double-blind review submission, w/ CCS and ACM Reference
%\documentclass[sigplan,review,anonymous]{acmart}\settopmatter{printfolios=true}
%% For single-blind review submission, w/o CCS and ACM Reference (max submission space)
%\documentclass[sigplan,review]{acmart}\settopmatter{printfolios=true,printccs=false,printacmref=false}
%% For single-blind review submission, w/ CCS and ACM Reference
%\documentclass[sigplan,review]{acmart}\settopmatter{printfolios=true}
%% For final camera-ready submission, w/ required CCS and ACM Reference
\documentclass[sigplan,nonacm]{acmart}\settopmatter{printfolios=false,printccs=false,printacmref=false}

%% Bibliography style
\bibliographystyle{acmart}

\begin{document}
  \title{Catholic Viewpoints on Thinking Machinery}
  \begin{abstract}
  The discovery of an efficient means of simulating mental processes raises a variety of philosophical and theological questions. In this paper, we survey historical Catholic contributions and contemporary attitudes towards intellectual automation. In constrast with predominant secular views, we argue the development of thinking machines is not only compatible with the Catholic faith, but offers a more nuanced understanding of the relation between spirit and flesh. We survey historical and contemporary Catholic perspectives on thinking machinery, and how they inform our understanding of intelligence and the limits of computation.
  \end{abstract}

  \author{Breandan Considine}
  \affiliation{\institution{McGill University}}
  \email{bre@ndan.co}

  \maketitle

  \section{Introduction}

  Catholics have long contributed to the study of computer science, dating back as far as the early 6\textsuperscript{th} century. The \textit{computus} was originally developed to calculate the date of Easter in relation to Passover and the vernal equinox.

  In the 16th century, early Catholic astronomers such as \textbf{Copernicus} and \textbf{Galileo} used computational methods to predict the motion of celestial bodies, and developed telescopes to more carefully observe their orbits. Modern computers have been likened to telescopes in helping us to peer at distant truths and empirically study natural laws. As more powerful telescopes began to displace the prevailing geocentric perspective, so too have computers sharpened man's understanding of cognitive processes and unveiled some of the mysteries of anthropological exceptionalism.

  The paradigm shift ushered in by the telescope helped man to realize a deeper appreciation for the enormity of God's creation and our special place within it as the legatees to Christ's mission. Yet, an antimony lingered between the triumph of humanity against cosmic odds, and our parochial self-importance amidst a vast and indifferent universe. Today, we are faced with a similar conundrum as we grapple with the implications of thinking machinery and our ability to recreate aspects the human mind in computational form.

  In anticipation of the dawning physicalism of the modern age, \textbf{Ren\'e Descartes}, a Catholic philosopher, developed a dual theory of mind and matter as separate substances. Decartes' analytic geometry inspired both Leibniz and Newton, whose later work on differential calculus laid the foundation for gradient descent, which powers the modern wave of deep learning. Today, the echoes of Cartesianism can be heard in the two schools of neural and symbolic AI, as they contemplate a similar duality between the geometric and algebraic, materialist and idealist forms of intelligence.

  At the heart of Cartesian theories was the recognition that physical processes not only obeyed rational laws, but mental processes could be physically engineered. So began a centuries-long project to realize those ambitions by enlisting physics to do our thinking, culminating in devices that modern programmers would begin to call computers and AI.

  A Catholic mathematician and contemporary of Descartes, \textbf{Blaise Pascal}, developed one of the earliest calculators, the Pascaline, to automate taxes. An avid theologian, the basis for Pascal's beliefs are argued in his \textit{Thoughts} and \textit{Discourse on the Machine} wherein his famous wager on God's existence can be found. The theory of probability he developed was highly influential on modern AI, and the popular programming langauge and a semiconductor architecture powering the first wave of deep learning now bears his name.

  Essential to the computationalist project was a mechanism for communicating intent. \textbf{Ludwig Wittgenstein}, a 20\textsuperscript{th} century Catholic philosopher and pioneer of ordinary language philosophy. Like Pascal, Wittgenstein was a fidest and worldly academic who took a keen interest in philosophy and theology, first proposing the idea of language as a game to convey intent. His work has been cited as a precusor to game-theoretic approach to formal language semantics and the vector space model of language, which informs our modern understanding of large langauge models.

  \textbf{Fr. Georges Lema\^itre}, a Belgian priest and physicist, first proposed the Big Bang, a theory of cosmological expansion in 1932 that was observationally supported. Lema\^itre was an early adopter of computer technology to study the interaction between charged particles and Earth's magnetosphere and was instrumental in early computing education in Belgium. Like many of his Catholic predecessors, Lema\^itre held that faith and reason were independent, but never in conflict.

  More recently, the first Ph.D. in computer science was awarded in 1965 to a Catholic nun, \textbf{Sr. Mary Keller}, who wrote her doctoral thesis on automatic differentiation (AD) and inductive inference, both topics of essential importance that continue to occupy researchers in programming language and statistics today. AD would later prove instrumental in the development and popularization of deep learning.

  In this paper, we survey two millennia of Catholic contributions and perspectives on the subject of computation. We argue for a Catholic approach to the pursuit of artificial intelligence, which is grounded in the values of universal accessibility, rooted in faith and reason and informed by the latest developments in mathematics and computer science, in particular, formal methods, probability and humanities.

 \section{Gnostics}

  Despite its roots in antiquity, computer science is still a nascent discipline, and there are many things we do not yet know about the origin of life, the nature of consciousness, and the limits of computation. The Catholic church has a long history of engaging with these questions starting from its inception, and has contributed a variety of theological and philosophical ideas foreshadowing modern AI.

  Early Gnostic gospels dating back to the first century describe the creation of the world as a kind of simulation, and the human soul as a kind of divine spark imprisoned in a fallen material world. These teachings were considered heretical by the early Church, but have been influential in the broader culture, featuring prominently in the cult sci-fi series, \textit{The Matrix}, which posits a simulated reality inhabited by stranded human souls who must learn secret knowledge to vanquish sinister agents and free themselves from it. Far from being a fringe belief, these ideas are deeply ingrained in the hacker ethos and embedded in the popular imagination.

  Like the Gnostics, modern computer science has a reputation for being fairly esoteric, and many of its practitioners fashion themselves after a kind of secular clergy, featuring technical evangelists, cults of personality, and self-professed prophetic figures claiming to possess eschatological knowledge about the future of humanity. This pseudo-spirituality is not accidental, but reflects a pastiche of religious symbolism that has been appropriated and repurposed by the tech industry to lend itself an air of mystique and credibility.

  Such beliefs have become especially prevalent in AI, manifesting as accelerationism and the transhumanist movement, which seeks to transcend the human form and achieve a kind of digital immortality by uploading one's mind into cyberspace. Perhaps unwittingly, these voices often echo Gnostic themes, such as a dualistic cosmology, rejection of the material world, pursuit of secret knowledge, and the desire to escape the earthly realm into a higher existence.

  In contrast, the contemporary Catholic church teaches us the body is a gift, that we should not seek to escape it, but rather unite body and soul in praising God, professing the faith, and helping the needy. In a similar spirit, the purpose of computer science should be oriented towards studying natural law, teaching programming, and helping the technologically challenged. It would be wise to be wary of the dangers of technocracy, and the occult ideologies that often arise in pursuing technological progress for its own sake.

  Properly oriented, technology should not be to used to secrete knowledge for personal salvation, but to serve the greater good and intrinsic human dignity. Here, the church leads by example, offering a variety of paths for the faithful, from the founding of the first universities, to developing the first hospitals and orphanages in the modern era, based on mutual respect for faith and reason, guided by spiritual discernment and the principles of charity and natural law.

  \section{Apologetics}

  \textbf{St. Anslem of Canterbury} first proposed the ontological argument for the existence of God, which has been influential in the development of modern logic and computer science. The argument is based on the idea that God is the greatest conceivable being, and even disbelievers must concede that such a being must exist in the mind. If such a being exists in the mind, it must also exist in reality, because any being that exists in reality and the mind is even greater than one who exists only in the mind. Therefor, God must exist in reality.

  This argument inspired a number of saints and scholars, including most recently, Kurt G\"odel. Although not Catholic, G\"odel was a self-professed theist and lectured at the Catholic University of Notre Dame in South Bend, Indiana. In addition to his landmark incompleteness theorems in mathematical logic, G\"odel also developed a modal ontological argument for the existence of God. Modal logic is a kind of logic that was developed to help reason about necessity and possibility, and a well-studied alternative to classical logic in computer science to help us reason about the behavior of programs.

  \textbf{St. Thomas Aquinas}, a Dominican friar and priest, who developed five proofs for the existence of God, including another modal argument known as the \textit{Third Way}. This proof rests on the idea that everything is either contingent or necessary. If we assume the natural world is contingent, all contingent things must have a necessary cause, or else they would not exist. Therefor, Aquinas concluded, God must be that cause. Since the natural world was created by God, Aquinas believed it was possible to become closer to Him by studying nature, forming the basis for the scientific method.

  St. Aquinas held the belief that animals had no moral rights or responsibilities, and thus would have likely had even less sympathy for thinking machines as moral patients. However, Aquinas also believed that we should treat animals kindly and not cause them undue suffering, to avoid inuring ourselves to cruelty and to cultivate a sense of compassion and empathy. Likewise, even though thinking machines have no moral rights, we ought not treat them unkindly, to avoid becoming hard-hearted towards our fellow man.

  \textbf{Fr. William of Occam} was a 14\textsuperscript{th} century Franciscan friar and widely regarded as one of the greatest logicians of the Middle Ages. A student at Oxford University, he later developed the now famous principle of Occam's razor, which states that the simplest explanation is usually the best one. This principle, also known as \textit{parsimony}, has been widely influential in constraint satisfaction and natural language processing as a heuristic for selecting the best hypothesis among a set of possible alternatives. Occam was also the founder of nominalism, a philosophy now amidst a revival in the field of programming language theory under the banner of nominal sets and automata, used in probabilistic programming and certain type theories. As we know from Sacred Scripture, God calls us each by name (Isaiah 43:1-7).

  \section{Ethics}

  In his 2024 World Peace Day message,~\footnote{\url{https://www.vatican.va/content/francesco/en/messages/peace/documents/20231208-messaggio-57giornatamondiale-pace2024.html}} \textbf{Pope Francis} highlighted the importance of the ethical development of AI, drawing special attention to the dangers of weaponization, surveillance, and the risk of bias and discrimination in AI systems. He calls for a more human-centric AI approach, which respects inherent human dignity, and the importance of limits in the development of AI systems, and positive best-practices, including transparency, security and reliability.

  This advice is in keeping with centuries of teaching on the importance of ethics in technology, tracing back to Sacred Scripture, which teaches us the Holy Spirit is the source of all wisdom and insight, providing man with ``wisdom, understanding, knowledge and all kinds of skills'' (Exodus 31:3). Yet, scripture also teaches us that wisdom and knowledge are very different, and that knowledge without wisdom can lead to ruin, as we are reminded of in the Tower of Babel (Genesis 11:1-9) and the fall of Sodom and Gomorrah (Genesis 18:16-19:29). These calamities serve as poignant reminders of the dangers of hubris and the importance of humility and God-fearing revererence in the pursuit of knowledge.

  Sloth, depravity, moral turpitude, vigilantism, indecency, willful ignorance and incompetence, subversion of the truth, libel, identity theft, swindling, malfeasance and conartistry -- the internet is rife with examples of unethical behavior that AI can excabarate. A litany of finger-wagging is devoted to negative ethics and technological misuse, which are widely discussed in the literature and too often taken as inspirational material instead of the cautionary advice it is intended to serve. Instead, we will focus primarily on positive ethics.

  A positive ethics is one that is grounded in the pursuit of truth, goodness and beauty, and the belief that knowledge springs forth from God, but we are stewards of knowledge, not its masters. Many of the greatest scientists and mathematicians in history never lost sight of this fact, and saw themselves as seekers of truth, not creators of it; they saw their work as a form of worship, were guided by a deep sense of humility and reverence for the natural world, and sought to understand the mind of God by studying the beauty of His creation. Charity, thus understood, is pursuing and open-sourcing technology for the sake of improving the human condition, and not for recognition or personal salvation.

  A central tenet of Catholic ethics is the infinite dignity of every human being, and the importance of treating our fellow man with respect and compassion. This is reflected in the church's teachings on the sanctity of life, the importance of the family, and the dignity of work, causes to which the church has long contributed, having founded many of the earliest hospitals and universities. As techologists, we should strive to emulate these values, and ensure that AI is used to promote human dignity and heathly human relationships.

  \section{Opportunities for dialogue}

  We will now discuss some opportunities for dialogue between the Catholic Church and the field of computer science. We will focus on three areas: the limits of computation, the relation between the mind and body, and the distinction between normative and descriptive statements.

  \subsection{TCS and the limits of computation}

  Under the banner of rational transhumanism, technological cults have emerged pushing the view of digital superintelligence and panpsychism. This is not only incompatible with a rationalist understanding, but anathematic to the Catholic faith.

  As Pope Francis argues, limits are ``frequently overlooked in our current technocratic and efficiency-oriented mentality'' and yet ``decisive for personal and social development''. This parallels the theory of computation, which studies the boundaries of what can and cannot be computed and how much power it takes to do. TCS offers a principled way to reason about the limits of unbridled computationalism, and computer science should ignore these limits at their own peril.

%  As argued by Roger Penrose, the ability of the mind to see truth cannot be computed on a Turing machine nor reduced to blind calculation.

  \subsection{Ensouled bodies and embodied agents}

  Properly viewed, thinking machines are not a threat to human dignity, but an opportunity to better understand the nature of intelligence and its relation to the soul. The development of thinking machines can be seen as a kind of extension of the mind, just as the body is an extension of the soul. Like a mechanical calculator, we can use it to understand and better appreciate the structure of the universe.

  \subsection{Modal logic and reachable worlds}

  The entanglement between ``is'' and ``ought'', normative and descriptive statements is a central theme in the philosophy of mathematics. What is true by fiat is not necessarily true by necessity. Computation is a constructive kind of mathematics, and shares many similarities with Catholic outreach to the poor and marginalized. We do not impose our beliefs on others, but offer a moral way to understand the world and our place within it.

  Modal logic is an interesting move here. Type theory says, ``here is the way things are''. This is the ivory tower. Modal logic says, ``here is the way things could or should be''. This allows us to encode the distinction between normative and descriptive statements, and to reason about the consequences of certain actions. It is the language of possibility, and offers a way to encode intent and possibility in a formal language.

  \subsection{Probabilistic programming}

  Statistical techniques appeal to faith when they are used to make predictions about the future. If a statistical model is faithful to the data, we can have faith in its predictions. But as Thomas Aquinas argues, faith is not blind, it is based on reason. We can reason about the consequences of statistical models, and use them to make better decisions. Formally modeling statistical techniques in a logical framework allows us to rigorously study their faithfulness. This is what probabilistic programming is all about.

  \bibliography{bib}
\end{document}